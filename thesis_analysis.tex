
\chapter{Sensitivity to aQGC in \textit{WWZ} Production} \label{Chapter4}
The simulation and generation of events is described in the previous chapter. This chapter details the phenomenological study carried out in this thesis and the respective figures obtained for aQGC operators (M5 operator in this instance) which is defined in Table \ref{Mixedtable}. The Section \ref{Event Selection} describes defining  physics objects involved and undertaking event selection to obtain three charged leptons (oppositely charged leptons $\ell^{+}\ell^{-}$) from \textit{Z} boson and (other lepton $\ell$ and neutrino $\nu$) from \textit{W} boson and two jets ($ \text{q}\overline{\text{q}}$) in the final state. The reconstruction of candidates for three bosons from selected events is outlined in Section \ref{Event Reconstruction}. Furthermore, the imposition of kinematic cuts on important observables and its impact is discussed in Sections \ref{KinematicCuts} and \ref{Kinematic Distributions}. Section \ref{Discriminating Variables} displays figures for discriminating variables. Also, a statistical test is performed to quantify the discriminating power of kinematic variables. 

The EFT Lagrangian (for a given energy scale $ \Lambda $) for aQGC operators includes terms from both SM and higher dimensional operators to the triple gauge coupling of the SM Lagrangian. Tri-boson (\textit{WWZ} for this study) production is a rare process as predicted by the Standard Model. It involves self-interaction among the bosons of weak interaction through triple and quartic gauge boson couplings (as specified in Section \ref{Theory-QGC}), which are sensitive to contributions from unknown particles and/or interactions. The objective of this study is to understand how the \textit{WWZ} production is affected by the inclusion of aQGC operators and to identify discriminating variables that will be useful for future analyses.

In this study, the samples generated without inclusion of aQGC operators are referred to as SM samples while those that are generated by including them are referred to by the name of respective operators. Hence, the aQGC samples include both SM and higher-dimensional operator components. %Furthermore, if any kinematic cut is placed during the generation of events, the respective sample is defined by including both the operator and the respective cut.

The values of deviations in cross-section for these higher-dimensional EFT operators provided in Section \ref{DeviationsTable},  indicate that the limits on scalar and tensor operators are tightly constrained, while there is further scope for constraining mixed operators. Currently, the M5 operator is chosen for undertaking this study, since its inclusion results in the largest deviations of cross-sections as shown in Table \ref{table2}. In the case of other operators, their structure (as defined in Section \ref{aQGC Operators}) and the deviations impacts the nature of distributions for various observables. Therefore, the same analysis is performed on other operators (apart from M5) with significant deviations and their behavior can be seen in Appendix \ref{AppendixA} for M2 and T5 operators (without descriptions).
% Appendix \ref{AppendixA} for M2 and T5 operators respectively.

%
\section{Event Selection}\label{Event Selection}
\texttt{Rivet} (introduced in Section \ref{Rivet}) is a tool to define physics objects like jets, leptons, neutrinos and photons, isolate them based on analysis strategy and to produce histograms for these observables. It is also possible to define cuts to make the objects suitable for our analysis.

The steps involved in the analysis are described  below. Also, there are a number of Rivet classes that are inherited from the Rivet name-space thereby enabling increased functionality and flexibility in developing analysis through the rivet routines.  The script for rivet routine and the list of Rivet classes inherited is referred in Appendix \ref{Scripts}.

\subsection{Object Definition}
 A "cut" \label{CutDefn}is a requirement placed on every event, based on (minimum or maximum values of) kinematic variables that needs to be fulfilled for the event to be accepted.  Else the event is rejected (vetoed). These cuts are useful to obtain events with the exact final states relevant for this study. "Baseline-cuts" are cuts placed on generic entities (like Jets, leptons, neutrinos) to be well-defined physics objects. These entities are the building blocks for obtaining the reconstructed candidates of bosons. It enables in reducing the number of fakes with the help of these cuts.

For leptons, a  $p_{T}>10 \text{ GeV}$ and $|\eta|<2.5$  are required, while for jets the corresponding cuts are $p_{T}>25 \text{ GeV}$ and $|\eta|<5.0$ are required. The tau decay to leptons is allowed and "leptons" are referred to as \texttt{electrons} and \texttt{muons} only for the purposes of this analysis. Jets are reconstructed from the $Antik_{T}$ jet clustering algorithm \cite{anti-ktjet} as per its implementation using FastJet \cite{FastJet} with a distance parameter of $R =0.4$. There are no constraints placed on neutrinos. The utility of imposing baseline-cuts is that it reduces smearing of the peaks of invariant masses and ensures good quality candidates for the reconstruction of bosons. 

\subsection{Jet \& Lepton Selection}\label{EventSelection}
\subsubsection{Jet Selection}
Every event is required to have at least one jet, else the event is vetoed. These jets are required to have $p_{T}>25 \text{ GeV}$ and the angular constraint of $|\eta|<5.0$ is also imposed in order to ensure that the jets are within the range of ATLAS detectors.

\subsubsection{Number of Jets \& Merged Jets}
The number of jets in each event is displayed in Figure \ref{NJets-PreCut} for both the M5 and SM samples. It is to be noted that this histogram is filled for events with all NJets.
\begin{figure}[!h]%{0.55\textwidth}
	\centering
	\includegraphics[width=0.8\textwidth]{/Users/keshavaprasad/Desktop/ThesisWriting/ThesisImages/Njets-PreCut.png}
	\caption{The number of jets after requiring at least one jet and $p_{T,jet}>25 \text{ GeV}$ for SM and M5 samples before imposing Jet selections, with only the statistical uncertainties being displayed.The events are normalized to one.}
	\label{NJets-PreCut}
\end{figure}
The former has a large number of events with a single jet while the latter has two or more jets in the majority of events. In certain cases, its possible that due to the boosted nature of the \textit{W} boson, that it decays into a pair of boosted jets and appears collimated due to the small opening angle between the two jets. This pair of boosted jets, being very closely clustered in the jet cone, is interpreted as "merged-jet" or "single-jet" by the $Antik_{T}$ algorithm for the M5 sample.

It is possible that the $Antik_{T}$ algorithm misinterprets a pair of jets as a  merged jet and could lead to faulty reconstruction of \textit{W} boson candidates. In order to verify this claim, the invariant mass for events with exactly one jet (which could be a merged jet) is obtained, which is shown in Figure \ref{Single Jet Mass-PreCut}.

 \begin{figure}[!h]
 	\centering
 	\includegraphics[width=0.8\textwidth]{/Users/keshavaprasad/Desktop/ThesisWriting/ThesisImages/singlejet_mass-PreCut.png}
 	\caption{The figure for single jet with invariant mass in range of \numrange[range-phrase = --]{50}{110} GeV after requiring $p_{T,jet}>25$ \text{ GeV} for SM and M5 samples, with only the statistical uncertainties being displayed. The last bin contains overflow. The events are normalized to one.}
 	\label{Single Jet Mass-PreCut}
 \end{figure}

 The SM sample has many single jets in the low mass region, while the peak at 80 GeV confirms that only the jets from boosted M5 sample is interpreted as single-jet by the $Antik_{T}$ algorithm. Also, the mass of single-jet is close to the mass of the \textit{W} boson (abbreviated as \texttt{$M_{W}$} with \texttt{$M_{W}=80.3$} \text{GeV}\cite{pdg2010}). Hence, those events with single jets which are not close to \texttt{$M_{W}$} are rejected. Therefore, for events with a single jet with invariant mass in range of \numrange[range-phrase = --]{50}{110} GeV, chosen as the candidate for the \textit{W} boson that decays hadronically for the reconstruction.

\subsubsection{Lepton Selection}
At least three charged leptons in the final state are required in every event else its is rejected. These leptons are obtained after applying the baseline cuts for all entities from the \texttt{dressedleptons} container that is provided by Rivet. The \texttt{dressedleptons} are bare leptons that are clustered together with photons \cite{DressedLeptons}.

The neutrinos could originate from the either a \textit{W} boson decay or from tau decays. These neutrinos are obtained from the \texttt{neutrinos} container provided by Rivet. Only one neutrino is present in the vast number of events. Therefore, only the neutrino with highest transverse momenta is utilized for reconstruction purposes in our analysis.

Exactly three leptons are present in the majority of events. However, additional leptons are observed in few instances (four or five leptons in certain events). This could be due to the presence of  virtual photons in the initial state radiation, decaying into leptons and thereby leading to additional number of leptons in the final state. Also, jets are mis-reconstructed as leptons in few instances.

%\begin{figure}[!h]%{0.55\textwidth}
%	\centering
%\includegraphics[width=0.8\textwidth]{/Users/keshavaprasad/Desktop/ThesisWriting/ThesisImages/Nlep-PreCut.png}
%	\caption{The plot displaying number of leptons before imposing Cuts}
%	\label{NLeps-PreCut}
%\end{figure}

In conclusion, only the objects that have passed the baseline-cuts are eligible to be defined as jets, leptons and neutrinos. Each event must contain at least one jet, a minimum of three charged leptons and at least one neutrino associated with it.

\subsection{Event Selection Summary}
A Cut is defined in section \ref{CutDefn}. The "Cut-Flow" shown in Figure \ref{Cut flow Histogram-PreCut} refers to number of events that remain after passing conditions imposed during event selections. The number of events remaining after each requirement is displayed in the three bins of the histogram. The first bin contains the total number of events without applying any conditions, the second bin contains the number of events after the jet selection (at least one jet and $p_{T}>25 \text{ GeV}$) and the last bin contains events that have passed the lepton selection (minimum of three leptons in every event) along with choosing the neutrino with highest transverse momenta.
\begin{figure}[!h]
	\centering
	\includegraphics[width=0.8\textwidth]{/Users/keshavaprasad/Desktop/ThesisWriting/ThesisImages/Ntotev-PreCut.png}
	\caption{The figure displays total number of events in the first bin, events passing jet constraints in the second bin and after passing lepton requirements in the third bin for SM and M5 samples, with the statistical uncertainties being displayed.}
	\label{Cut flow Histogram-PreCut}
\end{figure}

There are more \texttt{SM} events passing the jet selection criteria than the EFT operator while the opposite is true after lepton selection. In the M5 sample, bosons are boosted and thereby even the leptons from tau decays are boosted, which would easily satisfy the baseline cuts for jets. Hence, more leptons are found for the M5 sample than in the case of SM. In the case of jets, the baseline cuts were more stringent than leptons resulting in many jets that are low in transverse momenta being rejected. It results in fewer jets that pass for the M5 sample than SM.

\section{Event Reconstruction}\label{Event Reconstruction}
The procedure for reconstruction of the bosons from their decay products is outlined below for the candidate of each boson. The figures for the reconstructed candidates do not reveal any differences between the SM and aQGC samples. Hence, these figures are useful to verify that ancestors of decayed final state particles are indeed $W^{-}W^{+}Z$ bosons and also serves to prove the efficacy in the reconstruction mechanism.

\subsection{Hadronic \textit{W} Boson Candidate}
There are a number of events with two or more jets as seen in Figure \ref{NJets-PreCut}. The invariant mass for each jet pair is calculated and the candidate for \textit{W} boson that decays hadronically is chosen wherein the invariant mass of jet pair being closest in the vicinity of \texttt{$M_{W}$}. The figure displaying all possible invariant mass of candidates for the \textit{W} boson, from the pair of jets is shown in Figure  \ref{Had candidate-PreCut}.

\begin{figure}[!htb]%{0.8\textwidth}
	\centering
	\includegraphics[width=0.8\textwidth]{/Users/keshavaprasad/Desktop/ThesisWriting/ThesisImages/Whad_m_q_q-PreCut.png}
	\caption{The reconstructed invariant mass in range of \numrange[range-phrase = --]{50}{110} GeV of all possible candidates from pairs of jets, for SM and M5 samples with only the statistical uncertainties being displayed. The events are normalized to one.}
	\label{Had candidate-PreCut}
\end{figure}

\subsection{Leptonic \textit{Z} Boson Candidate}
 The leptons, which reconstruct the \textit{Z} boson, must qualify a precondition that they be of same flavor and opposite charge. The reconstruction is performed by summing over the four momenta of the constituent leptons of the \textit{Z} boson, with their invariant mass closest to mass of \textit{Z} boson.
\begin{figure}[!htb]%{0.65\textwidth}
	\centering
	\includegraphics[width=0.8\textwidth]{/Users/keshavaprasad/Desktop/ThesisWriting/ThesisImages/Z_m_l_l-PreCut.png}
	\caption{The reconstructed invariant mass of all possible  pairs of opposite-sign-same-flavored leptons close to mass of \textit{Z} boson, for SM and M5 samples with the statistical uncertainties being displayed. The last bin contains overflow. The events are normalized to one.}
	\label{Z Candidate Mass-PreCut}
\end{figure}
The \textit{Z} boson candidate is selected based on invariant mass criterion, such that the candidate's mass is closest to mass of \textit{Z} boson (abbreviated as \texttt{$M_{Z}$} with \texttt{$M_{Z}=91.2$} \text{ GeV} \cite{pdg2010}). This is similar to the case of reconstructed \textit{W} boson candidate using a pair of jets. The figure displaying the reconstruction of all possible \textit{Z} boson candidates is provided in Figure  \ref{Z Candidate Mass-PreCut}.

\subsection{Leptonic \textit{W} Boson Invariant Mass Candidate}
Based on the event selection specified earlier, after obtaining candidates for \textit{W} \& \textit{Z} bosons (that decays hadronically and leptonically respectively), the neutrino and a lone lepton remains unused so far for reconstruction. The \textit{W} boson candidate that decays leptonically is reconstructed by choosing remaining-lepton along with neutrino with hardest-$p_{T}$. The invariant mass for the candidate is checked to be closest to that of \texttt{$M_{W}$}. In every event, the remaining-lepton refers to the lepton(s) not involved in the reconstruction of \textit{Z} boson candidate. In certain instances when there are more than one remaining-leptons available, the lepton with hardest $p_{T}$ among them is selected for performing reconstruction of invariant mass candidate as displayed in Figure \ref{Lep Candidate-PreCut}.
\begin{figure}[!htb]%{0.8\textwidth}
	\centering
	\includegraphics[width=0.8\textwidth]{/Users/keshavaprasad/Desktop/ThesisWriting/ThesisImages/Wlep_m_l_nu-PreCut.png}
	\caption{The reconstructed invariant mass from remaining lepton with highest transverse momenta and hardest neutrino, for SM and M5 samples with only the statistical uncertainties being displayed. The last bin contains overflow. The events are normalized to one.}
	\label{Lep Candidate-PreCut}
\end{figure}

\section{Key Observables and Implementing Kinematic Cuts}\label{KinematicCuts}
The two key observables $H_{T}$ \& invariant mass of the entire system (abbreviated henceforth as \texttt{InvMass$\_$WWZ}) are defined below. Their importance is due to the fact that these observables are defined by including all the constituents of the system. This makes them suitable candidates for placing requirements and is advantageous in the context of this study, since it allows to impose a kinematic requirements on minimum number of observables and impacts all the constituents of the system. The procedure for generating different samples after placing these cuts is outlined in the following section. Furthermore, various kinematic observables impacted by these cuts are displayed.

\subsection{Key Observables}
The variable $H_{T}$ is defined as:
\begin{equation}
H_{T}  =  \sum_{\text{visible particles}} p_{T}
\end{equation} where the visible particles includes all jets and leptons decaying. It is to be noted that the contributions of neutrinos is not included. The histogram for $H_{T}$ is obtained and displayed in Figure \ref{HT-PreCut}.

The invariant mass of the entire system (abbreviated henceforth as \texttt{InvMass$\_$WWZ}) is obtained and is shown in Figure \ref{InvMass-WWZ-PreCut}. Since the \textit{W} boson candidate is reconstructed through contribution from neutrino with hardest-$p_{T}$, consequently the \texttt{InvMass$\_$WWZ} also includes contributions from the neutrinos as well and hence witnesses a large amount of overflow for the invariant mass of the system.
\begin{figure}[!htb]
	\centering
	\includegraphics[width=0.8\textwidth]{/Users/keshavaprasad/Desktop/ThesisWriting/ThesisImages/HT-PreCut.png}
	\caption{The figure for $H_{T}$ for SM and M5 samples, with the statistical uncertainties being displayed. The last bin contains overflow. The events are normalized to one.}
	\label{HT-PreCut}
\end{figure}

\begin{figure}[!htb]
	\centering
	\includegraphics[width=0.8\textwidth]{/Users/keshavaprasad/Desktop/ThesisWriting/ThesisImages/WWZ_m-PreCut.png}
	\caption{The invariant mass of the \textit{WWZ} system for SM and M5 samples, with the statistical uncertainties being displayed. The last bin contains overflow. The events are normalized to one.}
	\label{InvMass-WWZ-PreCut}
\end{figure}

\subsection{Implementing Kinematic Cuts}\label{Implementing Cuts}

As mentioned earlier, the EFT Lagrangian includes both SM and aQGC terms (M5 operator is defined in Table \ref{Mixedtable}) for the triple gauge coupling of the SM Lagrangian. In the M5 sample, the inclusion of such EFT operators is the cause for these bosons and its respective decay products to be boosted in their final states. This can be inferred from the high-$p_{T}$ tail of the M5 sample, which is shown up to 5 TeV in Figure \ref{HT-PreCut}. In the case of SM sample, it is not boosted in the high-$p_{T}$ region since it lacks EFT operators and does not extend its tail beyond 2500 GeV.

$H_{T}$ is an important observable for this analysis with different cuts being imposed on it to probe the validity of an effective theory (as described in section \ref{EFT-Validity}). The imposition of cuts also has a bearing on kinematics of other observables. The impact of cuts enables us to understand the energy scales at which these higher dimensional operators become relevant and facilitates in constraining the limits of these operators to high levels of precision. The cuts on $H_{T}$ also helps in identifying discriminating variables that could be exploited during experimental analysis to (dis)prove the existence of said process in EFT framework.

After placing the cuts on $H_{T}$, the rivet routine is run on the generated event samples (as described in \ref{EventGeneration} ) to obtain its output in terms of Root \cite{ROOT} files. The distributions for each observable, with different cuts, is displayed in the same histogram for comparison with the sample from \texttt{SM} in Figure \ref{HT}. These rivet-analyzed samples are defined on the basis of cut values in the Table \ref{table4}. 

 \begin{table}
 	\centering

 	\begin{tabular}{|c|c|c|c|}
 		\hline
 		Sample & Kinematic Cut & Color (In Figures) & Definition (in text)\\
 		\hline \hline
 		SM & No Cut & Blue & \texttt{SM}\\
 		\hline
 		M5 & No Cut & Black & \texttt{M5-NoCut} \\
 		\hline
 		M5 & 3 TeV & Red & \texttt{M5-3TeVCut}\\
 		\hline
 		M5 & 1 TeV & Orange & \texttt{M5-1TeVCut} \\
 		\hline

 		%\caption{The best limits for tensor operators was obtained from ~\cite{Sirunyan:2017fvv}}
 	\end{tabular}

 	%\end{tabular}
 	\caption{The table defining the samples after imposing kinematic cuts.}
 	\label{table4}
 \end{table}
 Although similar cuts were placed on the \texttt{SM} sample too, these samples are not included in the figures since they do not display a marked difference in its distributions with or without the cuts.

It could be argued that, since both $H_{T}$ and \texttt{InvMass$\_$WWZ} contain boosted final state entities, they are both well suited for placing kinematic cuts in order to check the validity of EFT. A two dimensional figure for \texttt{InvMass$\_$WWZ} against $H_{T}$ in Figure \ref{2D}, reveals a (nearly) linear correlation between the two variables. This implies that the cuts would be equally valid for either of these observables. However, the key difference between these two variables is that $H_{T}$ does not include contribution from neutrinos. This exclusion is important since the $p_{T}$ of neutrino cannot be measured in the ATLAS detectors \cite{ATLASDetector} at LHC. Therefore, it was decided that placing cuts on $H_{T}$ would be more useful for this analysis than on \texttt{InvMass$\_$WWZ}.
\begin{figure}[!htb]
	\centering
	\includegraphics[width=0.8\textwidth]{/Users/keshavaprasad/Desktop/ThesisWriting/ThesisImages/2d.png}
	\caption{The two dimensional plot for $H_{T}$ Vs \texttt{InvMass$\_$WWZ} for M5 sample}
	\label{2D}
\end{figure}
\begin{figure}[!htb]
	\centering
	\includegraphics[width=0.8\textwidth]{/Users/keshavaprasad/Desktop/ThesisWriting/ThesisImages/HT.png}
	\caption{$H_{T}$ for SM (in blue), \texttt{M5-NoCut} (in black), \texttt{M5-3TeVCut} (in red) and \texttt{M5-1TeVCut} (in orange) samples, with only the statistical uncertainties being displayed. The last bin contains the overflow. The events are normalized to one.}
	\label{HT}
\end{figure}

In the high-$p_{T}$ region, the \texttt{M5-NoCut} sample displays a long tail, while \texttt{SM} sample is non-existent in this region beyond $H_{T} = 3 $  \text{TeV}. This is the motivation behind first cut at $\Lambda= 3  \text{TeV}$, while the limits on operators mentioned in Table \ref{table2} is normalized by $\Lambda= 1 $ \text{TeV}, motivating the other cut. $\Lambda$ is the energy scale at which the validity of the EFT is being probed.

The \texttt{M5-3TeVCut} also displays high-$p_{T}$ tail, until its sharp cut off at $H_{T} = 3 $ \text{ TeV}. On the other hand there is a near-identical overlap in the low-$p_{T}$ region, between \texttt{SM} and \texttt{M5-1TeVCut}, including their large peaks in the range of 300 GeV to 400 GeV. This indicates that the aQGC samples dominates in the high-$p_{T}$ region while \texttt{SM} is dominant in the low-$p_{T}$ region, with large interference effects in the intermediate energy scales.

The extremely close overlap of  \texttt{SM} and \texttt{M5-1TeVCut} shows that $H_{T} = 1 \text{ TeV}$ proves to be a stringent cut on the effective theory and reduces the impact of aQGC operators on kinematics significantly and at this energy scale the effective theory is nearly identical to SM itself. It can be further understood that the boosting of the bosons due to the inclusion of EFT operators is nullified to a large extent by placing stringent cuts on EFT models and thereafter such EFT models are a good approximation to SM itself (since SM term is the LO term in the perturbative expansion of EFT).

\begin{figure}[!htb]
	\centering
	\includegraphics[width=0.8\textwidth]{/Users/keshavaprasad/Desktop/ThesisWriting/ThesisImages/WWZ_m.png}
	\caption{The invariant mass of the \textit{WWZ} system for SM (in blue), \texttt{M5-NoCut} (in black), \texttt{M5-3TeVCut} (in red) and \texttt{M5-1TeVCut} (in orange) samples, with the statistical uncertainties being displayed. The last bin contains the overflow. The events are normalized to one.}
	\label{InvMass-WWZ}
\end{figure}

\section{Replicated Figures - Post Kinematic Cuts}\label{Kinematic Distributions}

A number of kinematic distributions can be obtained from the candidates of three bosons, which are described below. These variables - as can be seen from their figures - together form a large set of discriminating observables that could be useful to indicate the path for future probes and provide a guide for experimental analyses of aQGC. It is important to identify such observables in order to achieve the main objective of this thesis: to understand how the $W^{-}W^{+}Z$ production is affected due to the presence of EFT operators. In the earlier Section \ref{Event Selection}) and Section \ref{Event Reconstruction}, only figures of samples without any cuts was displayed. 

The impact on kinematic observables due to the imposition of cuts, is seen by comparing both samples - with cuts and without cuts- in the same histogram. Therefore, these histograms are shown again in the following subsection by including the samples that also includes the cuts as defined in Table \ref{table4}. The nature and motivation for these cuts and definitions of samples is provided in Section \ref{Implementing Cuts}.

%\subsection{Replicated Histograms - Post Cuts}
With regard to the cut-flow histogram in Figure \ref{Cut flow Histogram-PostCuts}, all the M5 samples with cuts has reduced number of events after passing jet and lepton selection in comparison to \texttt{SM}. The \texttt{M5-1TeVCut} sample is the most affected with least number of jet and lepton events in comparison to other samples. This indicates that the kinematic cut affects not just the nature of distributions of observables but also event selections. There is no change in the number of leptons for both pre- and post-cuts for all M5 samples.

In the histogram for number of jets in Figure \ref{NJets-PostCuts}, the \texttt{M5-3TeVCut} and \texttt{M5-NoCut} samples have many single-jets compared to SM. This is due to the imperfect resolution of boosted jets by the $Anti-k_{T}$ algorithm. Also, nearly half of all events have precisely a pair of jets for \texttt{SM} while it is reduced for the boosted samples. The more boosted a sample is (like \texttt{M5-NoCut}), the  lesser number of events that have well resolved pair of jets.

From the invariant masses of single jets in Figure \ref{Single Jet Mass -Post Cuts},  it can be understood that the cuts results in smaller number of single jet events that could be candidates for \textit{W }boson after each successive cut on the M5 sample. This is the first indication that imposing cuts could impact on these boosted nature of EFT operators. This can be seen from similar behavior of \texttt{SM} and \texttt{M5-1TeVCut} samples. They dominate the low mass region, while the boosted samples are dominant at higher values, reconfirming that EFT operators causes boosting of jets. Few of these single jets could themselves be candidates for the hadronically decaying \textit{W} boson.

The reconstruction of boson candidates is described in Section \ref{Event Reconstruction}. The same procedure is repeated for reconstruction of samples of \texttt{SM}, \texttt{M5-1TeVCut},  \texttt{M5-3TeVCut} and \texttt{M5-NoCut} and displayed in Figures \ref{Had Candidate}, \ref{Z Candidate} and \ref{Lep Candidate}. There is no discernible difference between the distributions and hence reconstructed invariant masses will not be useful as variables for discriminating between aQGC and SM. The small peak close to zero seen in Figure \ref{Lep Candidate} is due to extremely low-$p_{T}$ neutrinos  since they do not need to qualify any baseline-cuts. 

\begin{figure}[!htb]
	\centering
	\includegraphics[width=0.8\textwidth]{/Users/keshavaprasad/Desktop/ThesisWriting/ThesisImages/Ntotev.png}
	\caption{The total number of events in the first bin, events passing jet constraints in the second bin and after lepton requirements in the third bin for \texttt{SM} (in blue), \texttt{M5-NoCut} (in black), \texttt{M5-3TeVCut} (in red) and \texttt{M5-1TeVCut} (in orange) samples, with only the statistical uncertainties being displayed.}
	\label{Cut flow Histogram-PostCuts}
\end{figure}

\begin{figure}[!htb]%{0.55\textwidth}
	\centering
	\includegraphics[width=0.8\textwidth]{/Users/keshavaprasad/Desktop/ThesisWriting/ThesisImages/Njets.png}
	\caption{The number of jets after requiring at least one jet and $p_{T,jet}>25 \text{ GeV}$  for \texttt{SM} (in blue), \texttt{M5-NoCut} (in black), \texttt{M5-3TeVCut} (in red) and \texttt{M5-1TeVCut} (in orange) samples, with only the statistical uncertainties being displayed. The events are normalized to one.}
	\label{NJets-PostCuts}
\end{figure}

\begin{figure}[!htb]
	\centering
	\includegraphics[width=0.8\textwidth]{/Users/keshavaprasad/Desktop/ThesisWriting/ThesisImages/singlejet_mass.png}
	\caption{The invariant mass of a single-jet after requiring $p_{T,jet}>25 \text{ GeV}$ for \texttt{SM} (in blue), \texttt{M5-NoCut} (in black), \texttt{M5-3TeVCut} (in red) and \texttt{M5-1TeVCut} (in orange) samples, with only the statistical uncertainties being displayed. The last bin contains overflow. The events are normalized to one.}
	\label{Single Jet Mass -Post Cuts}
\end{figure}

\begin{figure}[!htb]%{0.8\textwidth}
	\centering
	\includegraphics[width=0.8\textwidth]{/Users/keshavaprasad/Desktop/ThesisWriting/ThesisImages/Whad_m_q_q.png}
	\caption{The invariant mass of all possible pair of jets, for \texttt{SM} (in blue), \texttt{M5-NoCut} (in black), \texttt{M5-3TeVCut} (in red) and \texttt{M5-1TeVCut} (in orange) samples, with only the statistical uncertainties being displayed. The events are normalized to one.}
	\label{Had Candidate}
\end{figure}


\begin{figure}[!htb]%{0.65\textwidth}
	\centering
	\includegraphics[width=0.8\textwidth]{/Users/keshavaprasad/Desktop/ThesisWriting/ThesisImages/Z_m_l_l.png}
	\caption{The invariant mass of all possible pairs of opposite-sign-same-flavored leptons, for \texttt{SM} (in blue), \texttt{M5-NoCut} (in black), \texttt{M5-3TeVCut} (in red) and \texttt{M5-1TeVCut} (in orange) samples, with only the statistical uncertainties being displayed. The events are normalized to one.}
	\label{Z Candidate}
\end{figure}

\begin{figure}[!htb]%{0.8\textwidth}
	\centering
	\includegraphics[width=0.8\textwidth]{/Users/keshavaprasad/Desktop/ThesisWriting/ThesisImages/Wlep_m_l_nu.png}
	\caption{The invariant mass of the candidates from hardest remaining lepton and the hardest neutrino, for \texttt{SM} (in blue), \texttt{M5-NoCut} (in black), \texttt{M5-3TeVCut} (in red) and \texttt{M5-1TeVCut} (in orange) samples with only the statistical uncertainties being displayed. The events are normalized to one.}
	\label{Lep Candidate}
\end{figure}

\section{Discriminating Variables }\label{Discriminating Variables}
%\section{Kinematic Distributions}\label{Kinematic Distributions}
The identification of discriminating variables is extremely useful for future experimental analyses of aQGC operators. From the figures of invariant masses (in Figures \ref{Had Candidate}, \ref{Z Candidate} and \ref{Lep Candidate}), it can be seen that only tiniest of differences is observed in the behavior between \texttt{SM} and different \texttt{M5} samples. Therefore, invariant mass will be unsuitable as a discriminating variable for future analyses. 

Since the M5 (or other aQGC samples are boosted), the transverse momenta of the constituent objects and the differences in their angular values, is a very useful discriminant in this regard. Due to the effects of cuts on the EFT samples, the distributions of \texttt{M5-3TeVCut} and \texttt{M5-1TeVCut} have lower peaks and shorter $p_{T}$ tails and consequently will be much less discriminating than the \texttt{M5-NoCut} sample. Therefore, it would be beneficial to use samples without placing any cuts for comparing with SM, in order to obtain a large discrimination between the two samples. 

Though the discrimination is clearly visible for certain variables, a statistical test will be used (in Section \ref{KSTest}) to quantify this discrimination in order to identify the most discriminating variable for each sample that has been generated in this study.

\subsection{Transverse Momentum of Boson Candidates}
The reconstruction of candidates for the three bosons was mentioned in Section \ref{Event Reconstruction}. Once the candidate have been obtained, its possible to access a number of kinematic quantities including transverse momenta and angular variables. The bosons are boosted due to the inclusion of EFT operators. These bosons are reconstructed with different constituents. Therefore, combining both the features mentioned above, this provides a handle to understand the impact of EFT operators. These figures for the transverse momenta of candidates carries a large discriminating power for aQGC models against \texttt{SM}. The figures are displayed in Figures \ref{Whad-pT}, \ref{WLep-pT} and \ref{ZlepleppT}. 

The samples have shorter tails in the case of leptons as compared to jets. The cuts imposed on the samples reflect on their respective distributions. This can be clearly seen from the reduced overflow in last bin for\texttt{M5-3TeVCut} compared to \texttt{M5-NoCut} sample. The  \texttt{M5-1TeVCut} is significantly impacted from the cut as can be seen by its exponential decrease similar to the behavior of \texttt{SM}, apart from its tail being non-existent in the region beyond 1200 GeV.
\begin{figure}[!htb]
	\centering
	\includegraphics[width=0.8\textwidth]{/Users/keshavaprasad/Desktop/ThesisWriting/ThesisImages/Whad_pt_q_q.png}
	\caption{The transverse momenta of the candidates for hadronically decaying \textit{W} boson for \texttt{SM} (in blue), \texttt{M5-NoCut} (in black), \texttt{M5-3TeVCut} (in red) and \texttt{M5-1TeVCut} (in orange) samples, with only the statistical uncertainties being displayed. The last bin contains overflow. The events are normalized to one.}
	\label{Whad-pT}
\end{figure}

\begin{figure}[!htb]
	\centering
	\includegraphics[width=0.8\textwidth]{/Users/keshavaprasad/Desktop/ThesisWriting/ThesisImages/Wlep_pt_l_nu.png}
	\caption{The transverse momenta of the candidates for leptonically decaying \textit{W} boson for \texttt{SM} (in blue), \texttt{M5-NoCut} (in black), \texttt{M5-3TeVCut} (in red) and \texttt{M5-1TeVCut} (in orange) samples, with only the statistical uncertainties being displayed. The last bin contains overflow. The events are normalized to one.}
	\label{WLep-pT}
\end{figure}

\begin{figure}[!htb]
	\centering
	\includegraphics[width=0.8\textwidth]{/Users/keshavaprasad/Desktop/ThesisWriting/ThesisImages/Z_pt_l_l.png}
	\caption{The transverse momenta of the candidates for leptonically decaying \textit{Z} boson for \texttt{SM} (in blue), \texttt{M5-NoCut} (in black), \texttt{M5-3TeVCut} (in red) and \texttt{M5-1TeVCut} (in orange) samples, with only the statistical uncertainties being displayed. The last bin contains overflow. The events are normalized to one.}
	\label{ZlepleppT}
\end{figure}

\subsection{Transverse Momentum of Leading Objects}
An object (jet/lepton) with highest value of transverse momenta is referred to as a leading object. The distributions of such leading objects is useful to gain an understanding of the behavior of individual objects (that reconstruct boson candidates) after imposing the kinematic cuts. The leptons and jets are identified based on MC particle numbering scheme defined in Reference \cite{PDG-ID-Code}. The figures for transverse momenta of leading objects will also prove to be useful in discriminating against \texttt{SM}. 

The leading \texttt{electron}, leading \texttt{muon} and leading \texttt{jet} is displayed in Figures \ref{leadingelectron-pt}, \ref{leadingmuon-pt} and \ref{leadingjet-pT} respectively. These distributions reaffirm that the samples with cuts, mimics the trend as witnessed in Figure \ref{HT}. The high-$p_{T}$ tail is found to be longer in the case of leading jet as compared to the leading leptons. This also seen through the overflows for each figure, with larger overflows in the case of jets than leptons. This indicates that the kinematic cut affects leptons much more than jets. It must be noted that in the case of leptons, there are three charged leptons in every event. On the other hand, majority of events (nearly 60\% of events for \texttt{M5-NoCut} sample) have either single-jets or pair of jets. Therefore, statistically the leptons are more susceptible to be affected by kinematic cuts than jets. 
\begin{figure}[!htb]
	\centering
	\includegraphics[width=0.8\textwidth]{/Users/keshavaprasad/Desktop/ThesisWriting/ThesisImages/leadingjet_pt.png}
	\caption{The transverse momenta of the candidates for the leading jet for \texttt{SM} (in blue), \texttt{M5-NoCut} (in black), \texttt{M5-3TeVCut} (in red) and \texttt{M5-1TeVCut} (in orange) samples, with only the statistical uncertainties being displayed. The last bin contains overflow. The events are normalized to one.}
	\label{leadingjet-pT}
\end{figure}

\begin{figure}[!htb]
	\centering
	\includegraphics[width=0.8\textwidth]{/Users/keshavaprasad/Desktop/ThesisWriting/ThesisImages/leadingelectron_pt.png}
	\caption{The transverse momenta of the candidates for the leading electron for \texttt{SM} (in blue), \texttt{M5-NoCut} (in black), \texttt{M5-3TeVCut} (in red) and \texttt{M5-1TeVCut} (in orange) samples, with only the statistical uncertainties being displayed. The last bin contains overflow. The events are normalized to one.}
	\label{leadingelectron-pt}
\end{figure}

\begin{figure}[!htb]
	\centering
	\includegraphics[width=0.8\textwidth]{/Users/keshavaprasad/Desktop/ThesisWriting/ThesisImages/leadingmuon_pt.png}
	\caption{The transverse momenta of the candidates for the leading muon for \texttt{SM} (in blue), \texttt{M5-NoCut} (in black), \texttt{M5-3TeVCut} (in red) and \texttt{M5-1TeVCut} (in orange) samples, with only the statistical uncertainties being displayed. The last bin contains overflow. The events are normalized to one.}
	\label{leadingmuon-pt}
\end{figure}

\subsubsection{Missing Transverse Energy}
The Missing transverse Energy ($E^{Miss}_{T}$ ) quantifies the missing energy of invisible particles that cannot be detected. It is displayed in Figure \ref{MET}. The behavior of M5 samples continues to follow the same pattern as earlier, with slightly lower overflows for \texttt{M5-3TeVCut} than \texttt{M5-NoCut}. Also, the \texttt{M5-1TeVCut} sample behaves similar to \texttt{SM} with slightly longer tail in transverse momenta.
\begin{figure}[!htb]
	\centering
	\includegraphics[width=0.8\textwidth]{/Users/keshavaprasad/Desktop/ThesisWriting/ThesisImages/MET.png}
	\caption{The missing transverse momentum for \texttt{SM} (in blue), \texttt{M5-NoCut} (in black), \texttt{M5-3TeVCut} (in red) and \texttt{M5-1TeVCut} (in orange) samples, with only the statistical uncertainties being displayed. The last bin contains overflow. The events are normalized to one. }
	\label{MET}
\end{figure}

\subsection{Angular Variables}
Similar to the transverse momenta of the candidates of three bosons, the angular variables ($\Phi$, $\eta$ and $\Delta R$  defined in References \cite{Sahoo:2016hln}, \cite{PhiEtaPt}, \cite{CDF-Coordinates}, \cite{LHC-Physics-Bristol}, \cite{Accelerators}, \cite{Baden-Kinematics}) of individual jets and individual leptons that are involved in the reconstruction of bosons, are also useful for their discriminating power against \texttt{SM}.

%\subsubsection{Azimuthal Angle ($\phi$)}
The angle ($\phi$) is the angle at which $p_{T}$ is measured in collider coordinates. Hence, the values of $\phi$ would directly impact the values obtained for $p_{T}$ for the given system. For example, in the case of two-body decay like a $\rightarrow$ b + c, its useful to obtain the difference in $\Phi$ for final state particles "b" and "c".

The $\Delta\phi$ is obtained by taking the difference in $\phi$ for constituent objects that were utilized in the reconstruction of respective candidates for bosons. The Figure \ref{Whad-dPhi-PostCuts} displays $\Delta\phi$ between the pair of jets that constitute the hadronically decaying \textit{W} boson candidate, between the pair of leptons decaying from \textit{Z} boson in the  Figure \ref{ZleplepdPhi-PostCuts} and in Figure \ref{WLepMeT-dPhi-PostCuts} for the remaining lepton and $E^{Miss}_{T}$ for leptonically decaying \textit{W} boson.

There are larger number of events at $\Delta\phi=0$ for \texttt{M5-3TeVCut} and \texttt{M5-NoCut} samples. With $\Delta\phi=0$ (between pair of jets, between pair of opposite sign leptons and finally between the other lepton and $E^{Miss}_{T}$), results in these objects being very close to each other in the transverse plane as they are boosted. The impact of cuts can be seen from the fact that the peak at $\Delta\phi=0$ is slightly reduced for \texttt{M5-3TeVCut} and is non-existent for \texttt{M5-1TeVCut}. Due to the large gap at $\Delta\phi=0$ when leptons are involved as compared to jets, the variables $\Delta\phi(l,\text{MET})$ and $\Delta\phi(l,l)$ turns out to be a better discriminant than $\Delta\phi(j,j)$.
\begin{figure}[!htb]%{0.65\textwidth}
\centering
\includegraphics[width=0.8\textwidth]{/Users/keshavaprasad/Desktop/ThesisWriting/ThesisImages/Whad_dPhi_q_q.png}
\caption{The $\Delta\phi$ between pair of jets for \texttt{SM} (in blue), \texttt{M5-NoCut} (in black), \texttt{M5-3TeVCut} (in red) and \texttt{M5-1TeVCut} (in orange) samples, with only the statistical uncertainties being displayed. The events are normalized to one.}
\label{Whad-dPhi-PostCuts}
\end{figure}
	%\centering
\begin{figure}[!htb]%{0.65\textwidth}
\centering
\includegraphics[width=0.8\textwidth]{/Users/keshavaprasad/Desktop/ThesisWriting/ThesisImages/Wlep_dPhi_l_MET.png}
\caption{The $\Delta\phi$ between between remaining lepton and $E^{Miss}_{T}$  for \texttt{SM} (in blue), \texttt{M5-NoCut} (in black), \texttt{M5-3TeVCut} (in red) and \texttt{M5-1TeVCut} (in orange) samples, with only the statistical uncertainties being displayed. The events are normalized to one.}
\label{WLepMeT-dPhi-PostCuts}
\end{figure}
	%\centering
\begin{figure}[!htb]%{0.65\textwidth}
\centering
\includegraphics[width=0.8\textwidth]{/Users/keshavaprasad/Desktop/ThesisWriting/ThesisImages/Z_dPhi_l_l.png}
\caption{The $\Delta\phi$ between between pair of opposite sign same flavor leptons for \texttt{SM} (in blue), \texttt{M5-NoCut} (in black), \texttt{M5-3TeVCut} (in red) and \texttt{M5-1TeVCut} (in orange) samples, with only the statistical uncertainties being displayed. The events are normalized to one.}
\label{ZleplepdPhi-PostCuts}
\end{figure}


%\subsubsection{Pseudo-Rapidity}
The Pseudo-Rapidity($|\eta|$) is one of the spatial coordinates in ($\eta$-$\phi$) space describing the angle of a particle relative to the beam axis. Pseudo-Rapidity($|\eta|$) is preferred over the polar angle ($\theta$) in collider physics and $\Delta|\eta|$ is obtained by taking the difference in $|\eta|$ for constituent objects of constructing bosonic candidates. The Figure \ref{Whad-dEta-PostCuts} displays $\Delta\eta$ between the pair of jets that constitute the hadronically decaying \textit{W} boson candidate, between the pair of leptons decaying from \textit{Z} boson in Figures \ref{ZleplepdEta-PostCuts}, \ref{WLep-dEta-PostCuts} for the remaining lepton and $\nu$ for leptonically decaying \textit{W} boson respectively.

If $\theta=0$, the particle is moving along the beam axis, and hence particles with high $|\eta|$ values are generally lost, escaping in the detector along with the beam. There are a large number of events at $\Delta|\eta|=0$ for \texttt{M5-3TeVCut} and \texttt{M5-NoCut} samples. However, \texttt{SM} decays exponentially and overlaps with \texttt{M5-1TeVCut}. The impact of cuts can be seen from the fact that the peak at $\Delta\Phi=0$ is marginally reduced for \texttt{M5-3TeVCut} and significantly reduced for \texttt{M5-1TeVCut}. Similar to the case of $\Delta\Phi$, the large gap at $\Delta|\eta|=0$ when leptons are involved as compared to jets, the variables $|\Delta\eta|(l,\nu)$ and $|\Delta\eta|(l,l)$ turns out to be a better discriminant than $|\Delta\eta|(j,j)$.


\begin{figure}[!htb]%{0.8\textwidth}
\centering
\includegraphics[width=0.8\textwidth]{/Users/keshavaprasad/Desktop/ThesisWriting/ThesisImages/Whad_dEta_q_q.png}
\caption{The $\Delta\eta$ between pair of jets for \texttt{SM} (in blue), \texttt{M5-NoCut} (in black), \texttt{M5-3TeVCut} (in red) and \texttt{M5-1TeVCut} (in orange) samples, with only the statistical uncertainties being displayed. The events are normalized to one.}
\label{Whad-dEta-PostCuts}
\end{figure}

\begin{figure}[!htb]%{0.8\textwidth}
	\centering
	\includegraphics[width=0.8\textwidth]{/Users/keshavaprasad/Desktop/ThesisWriting/ThesisImages/Wlep_dEta_l_nu.png}
	\caption{The $\Delta\eta$ between between remaining lepton and neutrino (with hardest $p_{T}$) for \texttt{SM} (in blue), \texttt{M5-NoCut} (in black), \texttt{M5-3TeVCut} (in red) and \texttt{M5-1TeVCut} (in orange) samples, with only the statistical uncertainties being displayed. The events are normalized to one.}
	\label{WLep-dEta-PostCuts}
\end{figure}

\begin{figure}[!htb]%{0.8\textwidth}
		\centering
		\includegraphics[width=0.8\textwidth]{/Users/keshavaprasad/Desktop/ThesisWriting/ThesisImages/Z_dEta_l_l.png}
		\caption{The $\Delta\eta$ between between pair of opposite sign same flavor leptons for \texttt{SM} (in blue), \texttt{M5-NoCut} (in black), \texttt{M5-3TeVCut} (in red) and \texttt{M5-1TeVCut} (in orange) samples, with only the statistical uncertainties being displayed. The events are normalized to one.}
		\label{ZleplepdEta-PostCuts}
\end{figure}

%\subsubsection{Delta R ($\Delta \text{R}$)}
The $\Delta \text{R}$ is defined to be : 
\begin{equation}
\Delta \text{R}  = \sqrt{(\Delta\phi)^2+(\Delta\eta)^2}
\end{equation} is useful to obtain a measure of the distance between objects in three-dimensional space is shown in Figures \ref{Whad-dR-PostCuts},  \ref{WLep-dR-PostCuts}, \ref{Z-dR-PostCuts}. The $\Delta \text{R}$ parameter is a representation of combination of effects from both angular variables. Due to the imperfect resolution of boosted jets, the jets more closed spaced and (likely) to be moving in the same direction, as can be confirmed from Figure \ref{Whad-dR-PostCuts}, with low values of separation between the jets. However, suffer no issues in its resolution and continues to be clearly discriminating due to its large discriminant across the entire range of values. 

	\begin{figure}[!htb]%{0.8\textwidth}
		\centering
		\includegraphics[width=0.8\textwidth]{/Users/keshavaprasad/Desktop/ThesisWriting/ThesisImages/Whad_dR_q_q.png}
		\caption{The $\Delta \text{R}$ between pair of jets for \texttt{SM} (in blue), \texttt{M5-NoCut} (in black), \texttt{M5-3TeVCut} (in red) and \texttt{M5-1TeVCut} (in orange) samples, with only the statistical uncertainties being displayed. The events are normalized to one.}
		\label{Whad-dR-PostCuts}
	\end{figure}
%	\centering
	\begin{figure}[!h]%{0.8\textwidth}
		\centering
		\includegraphics[width=0.8\textwidth]{/Users/keshavaprasad/Desktop/ThesisWriting/ThesisImages/Wlep_dR_l_nu.png}
		\caption{The $\Delta \text{R}$  between between remaining lepton and neutrino (with hardest $p_{T}$ ) for \texttt{SM} (in blue), \texttt{M5-NoCut} (in black), \texttt{M5-3TeVCut} (in red) and \texttt{M5-1TeVCut} (in orange) samples, with only the statistical uncertainties being displayed. The events are normalized to one.}
		\label{WLep-dR-PostCuts}
	\end{figure}

	\begin{figure}[!h]%{0.8\textwidth}
		\centering
		\includegraphics[width=0.8\textwidth]{/Users/keshavaprasad/Desktop/ThesisWriting/ThesisImages/Z_dR_l_l.png}
		\caption{The $\Delta \text{R}$ between between pair of opposite sign same flavor leptons for \texttt{SM} (in blue), \texttt{M5-NoCut} (in black), \texttt{M5-3TeVCut} (in red) and \texttt{M5-1TeVCut} (in orange) samples, with only the statistical uncertainties being displayed. The events are normalized to one.}
		\label{Z-dR-PostCuts}
	\end{figure}

\subsection{Kolmogorov Smirnov Test}\label{KSTest}
The Komogorov-Smirnov (KS) test \cite{KSTest}, \cite{KSTest2} is a statistical test useful for determining the compatibility in shape, between the two distributions. It can be inferred from the earlier subsections that, the \texttt{M5-NoCut} sample offers the largest discriminating power among the three samples, for a number of kinematic variables displayed so far. The objective of performing the KS test is to identify kinematic variable for each sample that offers largest discriminant in comparison to the \texttt{SM} sample. 

By performing the KS test between two figures, the test returns a value (by default) between 0--1, which is the probability of test (much less than one would mean that the figures are not compatible). Similarly, it also possible to obtain the "Kolmogorov Distance" between the two figures, which is the more relevant quantity for this study. It is especially useful to quantify the distance (i.e. non-compatibility or non-overlap) between the two distributions and thereby helps in identifying the most suitable discriminating variable, for a given sample of events. Therefore, the most discriminating variable has the largest values of the KS score (Kolmogorov Distance). 

The values of Kolmogorov distance is obtained for \texttt{M5-NoCut}, \texttt{M5-3TeVCut} and \texttt{M5-1TeVCut} samples in comparison to \texttt{SM}, enlisted in Table \ref{KS-Table1} for all potential discriminating variables. The transverse momenta of the \textit{Z} boson candidate is the best discriminant among all the variables available for analysis in the system for boosted samples. 

Among the angular variables, the $\Delta|\phi|$ between the pair of opposite sign same flavor leptons is the most discriminating variable, while the $\Delta|\eta|$ between pair of jets is not very well suited for the purpose of discriminating between SM and aQGC. Therefore, it can be concluded that leptons (which reconstruct \textit{Z} boson) offer best discrimination among all variables than other objects. 

The \texttt{M5-NoCut} sample offers the largest discriminant while comparing distributions. On the other hand, \texttt{M5-1TeVCut} is unsuitable for identifying discriminating variables. It must also be noted that though the KS test provides a means for quantifying the discriminating variables, only the values of KS scores will not be sufficient to conclude about the discriminating power of a given variable. It is important to ensure that the respective variables are also reconstructed efficiently.  

\begin{table}
		\centering
	\begin{tabular}{|c|c|c|c|}
		\hline
		Variable (Reference) &\texttt{M5-NoCut}&\texttt{M5-3TeVCut}&\texttt{M5-1TeVCut} \\
		\hline 
		\hline
		\textbf{Transverse Momenta of Candidates of Bosons}   &  && \\
		\hline
		Whad\_pt\_q\_q (refer to Figure  \ref{Whad-pT} ) & 0.398 & 0.333&0.027 \\
		\hline
		Wlep\_pt\_l\_nu   (refer to Figure  \ref{WLep-pT} ) & 0.551 & 0.453&0.030  \\
		\hline
		Z\_pt\_l\_l  (refer to Figure  \ref{ZlepleppT})  &\textcolor{red}{0.555} & \textcolor{red}{0.455} & 0.043 \\
		\hline
		\textbf{ Transverse Momenta of Leading Objects} &      &&\\
		\hline
		leadingjet\_pt (refer to Figure  \ref{leadingjet-pT} )& 0.506 &0.388 & 0.037  \\
		\hline
		leadingelectron\_pt (refer to Figure  \ref{leadingelectron-pt} ) & \textcolor{red}{0.513} &  \textcolor{red}{0.413} &0.018 \\
		\hline
		leadingmuon\_pt  (refer to Figure  \ref{leadingmuon-pt} ) & 0.505 & 0.399 & 0.033 \\
		\hline
		\textbf{ $E^{Miss}_{T}$} (refer to Figure \ref{MET}) & 0.457 & 0.387& 0.030  \\
		\hline
		\textbf{Angular Variables}  &    && \\
		\hline
		\textbf{$\Delta|\phi|$ between objects}   &  && \\
		\hline
		Whad\_dPhi\_q\_q  (refer to Figure  \ref{Whad-dPhi-PostCuts} ) & 0.205 & 0.177&0.038  \\
		\hline
		Wlep\_dPhi\_l\_MET (refer to Figure  \ref{WLepMeT-dPhi-PostCuts} )& 0.443 & 0.356&0.013 \\
		\hline
		Z\_dPhi\_l\_l  (refer to Figure  \ref{ZleplepdPhi-PostCuts} )& \textcolor{red}{0.489} & \textcolor{red}{0.396} & 0.024 \\
		\hline
		\textbf{$\Delta|\eta|$ between objects}&  && \\
		\hline
		Whad\_dEta\_q\_q (refer to Figure  \ref{Whad-dEta-PostCuts} )& 0.102& 0.086 & 0.007  \\
		\hline
		Wlep\_dEta\_l\_nu (refer to Figure  \ref{WLep-dEta-PostCuts} )& 0.371& 0.305&0.023 \\
		\hline
		Z\_dEta\_l\_nu  (refer to Figure  \ref{ZleplepdEta-PostCuts} )& \textcolor{red}{0.389} & \textcolor{red}{0.314} &0.034 \\
		\hline
	\end{tabular}
\caption{The table displaying values of Kolmogorov distance for variables associated with transverse momenta and angular variables of objects for \texttt{M5-NoCut}, \texttt{M5-3TeVCut} and \texttt{M5-1TeVCut} samples in comparison to \texttt{SM}. The most discriminating variable is highlighted. }
\label{KS-Table1}
\end{table}


%At the end of the introduction it is normal to say briefly what comes
%in the following chapters.
%
%The line at the beginning of this file is used by TeXstudio etc.\ to
%specify which is the master \LaTeX{} file, so that you can compile your thesis
%directly from this file.
%The lines at the end of this file are used by AUCTeX
%directly within \texttt{emacs} to do the same thing.
%If your thesis is called something other than \texttt{mythesis}, adjust them as appropriate.
%
%%%% Local Variables:
%%%% mode: latex
%%%% TeX-master: "mythesis"
%%%% End:
