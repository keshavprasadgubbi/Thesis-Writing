 %!TEX root = Keshav-aQGCThesis.tex

%==============================================================================
\chapter{Introduction}
\label{sec:intro}
%==============================================================================
\epigraph{I shall be telling this with a sigh\\
	Somewhere ages and ages hence:\\
	Two roads diverged in a wood, and I—\\
	I took the one less traveled by,\\
	And that has made all the difference}{\textit{"The Road Not Taken" by Robert Frost \cite{Frost}}}


Similar to the proverbial less-traveled path chosen by Frost, this study as part of the Master Thesis is the culmination of saga over past few years across all aspects of my life -- emotional, personal and professional. It is the first step (albeit small one)  towards the desire to study physics and contribute as a researcher in the this quest to understand how the universe works and to answer the question : "What is the universe made of?".

The quest to comprehend the mysterious universe is an ancient one, from the times of Aryabhatta \cite{Aryabhatta}, has continued over two past two millenniums. The progress has been accumulated in systematic and incremental steps, and has led to the development of "The Standard Model of Particle Physics" \cite{Thomson}. It is a theory (and has been confirmed through experimental evidences) that describes the forces and their interactions in our universe, although gravity has not yet been incorporated in Standard Model. 

The massive vector bosons are the mediators of Electro-Weak interactions, and form the focus of this study. They be either charged (\textit{$W^{\pm} $}) or neutral ($ \textit{Z}^{0} $), and are associated with several important processes of SM, and are focus of active research till this day. The production of three such bosons, was predicted by SM and the evidence for massive tri-boson production was obtained recently \cite{Aad:2019udh} with 4$ \sigma$ standard deviations. 

There have been numerous theoretical and experimental efforts in past three decades, to search for new physics beyond the predictions of Standard Model. The numerous observables in Standard Model can be probed for deviations, to search for indications of new physics. The operators of Effective Field Theory are be utilized to interpret the deviations from Standard Model predictions, with the coupling constants being free parameters of the "Effective" theory. 

A broad objective of this thesis is to perform a phenomenological study to understand the impact of anomalous Quartic Gauge Coupling operators for the production of \textit{WWZ} bosons at tree level. The conclusions/insights gained from this study will be useful for future experimental analyses of Quartic Gauge Coupling operators.

The chapters of the thesis are organized as follows: the theoretical underpinnings of self-coupling of Electro-Weak bosons and utility of probing for new physics in Effective Field Theory framework is provided in Chapter \ref{Ch2}. The procedure for production of \textit{WWZ} events at generator level and probe for deviations in cross-sections of operators is described in Chapter \ref{Ch3}. The selection of events, reconstruction of boson candidates, probing for validity of EFT models at different energy scales and identification of discriminating variables is described in Chapter \ref{Chapter4}. Lastly, a number of additional figures for different operators has been included in Appendix \ref{AppendixA}.  
%%% Local Variables: 
%%% mode: latex
%%% TeX-master: "mythesis"
%%% End: 
