%% !TEX root = Keshav-aQGCThesis.tex
%
\chapter{Conclusion \& Outlook}

A sensitivity study for \textit{WWZ} production by including aQGC operators was carried out in this phenomenological study and the observations and conclusions from the it is described below. 

The \textit{WWZ} events (at LO) were generated using \texttt{MadGraph5} event generator for different aQGC operators in the three lepton channel. The $\sigma_\text{SM}$ of WWZ production was found to be 89 fb. The most constrained limits for number of operators was compiled from different analyses; these limits were utilized to obtain the $\sigma_\text{aQGC}$ and their deviations were calculated in comparison to $\sigma_\text{SM}$. M5 operator was found to have the maximum deviation with respect to SM and was chosen for further phenomenological study. 

Event selection and reconstruction of boson candidates was carried out based on the invariant mass criterion. The inclusion of EFT operators causes boosting of the bosons for EFT samples. The $Antik_{T}$ algorithm is not suitable for resolution of jets, obtained from hadronically decaying \textit{W} boson. Therefore, it would be beneficial to use an alternative algorithm for future analyses. The validity of EFT power expansion is probed by placing kinematic cuts for the values of $\Lambda= 3$ \text{TeV} and $\Lambda= 1$ \text{TeV}. The net effect of placing cuts is that the boosting of bosons (due to the inclusion of EFT operators) is nullified. The 1 TeV cut proves to be an stringent cut and reduces the EFT sample to be a close approximation of SM itself. 

In the efforts towards identify discriminating variables, it was found that the samples without imposing cuts should be used as they offer the largest discriminant while comparing distributions. Also, invariant masses of candidates is not useful as an discriminating variable; Transverse momenta and angular variables are more effective in this regard. The transverse momenta of boson candidates in general, and specifically the $p_{T}$ of \textit{Z} boson candidate is found to be the most discriminating variable. KS test is employed to quantify the discrimination (through the values of Kolmogorov distance by comparing with SM) of different variables for a given sample. It is confirmed through the KS test that variables related to \textit{Z} boson offer the best discrimination in the case of M5 operator for both samples - with and without cuts.

The detector simulation could be included and operators could be further constrained. Currently, only one operator is switched on for generating samples of the particular aQGC operator. In order to probe interference effects, multiple operators could be switched on simultaneously. 

%%==============================================================================
%\chapter{Introduction}
%\label{sec:intro}
%%==============================================================================
%
%The introduction usually gives a few pages of introduction to the
%whole subject, maybe even starting with the Greeks.
%
%For more information on \LaTeX{} and the packages that are available
%see for example the books of Kopka~\cite{kopka04} and Goossens et
%al~\cite{goossens04}.
%
%A lot of useful information on particle physics can be found in the
%\enquote{Particle Data Book}~\cite{pdg2010}.
%
%I have resisted the temptation to put a lot of definitions into the
%file \texttt{thesis\_defs.sty}, as everyone has their own taste as
%to what scheme they want to use for names.
%However, a few examples are included to help you get started:
%\begin{itemize}
%\setlength{\itemsep}{0pt}\setlength{\parskip}{0pt}
%\item cross-sections are measured in \si{\pb} and integrated
%  luminosity in \si{\invpb};
%\item the \KoS is an interesting particle;
%\item the missing transverse momentum, \pTmiss, is often called
%  missing transverse energy, even though it is calculated using a vector sum.
%\end{itemize}
%Note that the examples of units assume that you are using the
%\textsf{siunitx} package.
%
%It also is probably a good idea to include a few well formatted
%references in the thesis skeleton. More detailed suggestions on what
%citation types to use can be found in the \enquote{Thesis Guide}~\cite{thesis-guide}:
%\begin{itemize}
%\item articles in refereed journals~\cite{pdg2010,Aad:2010ey};
%\item a book~\cite{Halzen:1984mc};
%\item a PhD thesis~\cite{tlodd:2012} and a Diplom thesis~\cite{mergelmeyer:2011};
%\item a collection of articles~\cite{lhc:vol1};
%\item a conference note~\cite{ATLAS-CONF-2011-008};
%\item a preprint~\cite{atlas:perf:2009} (you can also use
%  \texttt{@online} or \texttt{@booklet} for such things);
%\item something that is only available online~\cite{thesis-guide}.
%\end{itemize}
%
%At the end of the introduction it is normal to say briefly what comes
%in the following chapters.
%
%The line at the beginning of this file is used by TeXstudio etc.\ to
%specify which is the master \LaTeX{} file, so that you can compile your thesis
%directly from this file.
%The lines at the end of this file are used by AUCTeX
%directly within \texttt{emacs} to do the same thing.
%If your thesis is called something other than \texttt{mythesis}, adjust them as appropriate.
%
%%%% Local Variables: 
%%%% mode: latex
%%%% TeX-master: "mythesis"
%%%% End: 
