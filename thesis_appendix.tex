%------------------------------------------------------------------------------
\chapter{Additional Figures}\label{AppendixA}


\section{Matching Jets and Leptons for M5 Operator}
%\subsubsection{pT of Matching jets }
The procedure for identifying and obtaining matching jets and matching leptons are very similar to each other. From the array of jets or leptons, the matching objects are obtained through the index of the lepton or the jet, that is being utilized for reconstructing the best candidate in each iteration for the candidate of respective boson.  

The transverse momenta of these individual jets are shown in Figures \ref{Jet1_pt}, \ref{Jet2-pt}. From the invariant mass of the individual jets matched to hadronically decaying \textit{W} boson, as displayed in Figures \ref{Jet1-Mass} and \ref{Jet2_Mass}. It is clear that the boosting of jets in EFT samples due to the inclusion of EFT operators causes imperfect resolution of jets by the $Antik_{T}$ algorithm. Therefore, it is necessary to prevent contamination of such low-$p_{T}$ jets during the reconstruction of \textit{W} boson candidate. This was reason behind a nuanced and stringent during both jet selection and reconstruction of \textit{W} boson. Alternatively, it would be beneficial to use a different algorithm than $Antik_{T}$ for resolution of jets. 

The transverse momenta of the matching same flavor and opposite sign leptons are shown in Figures \ref{Zlep1_pt} and \ref{Zlep2-pt}. The transverse momenta of the remaining lepton matched to leptonically decaying \textit{W} boson is shown in Figure \ref{Wlep-pT}. 
%\begin{figure}

\begin{figure}[!htb]%{0.65\textwidth}
	\centering
	\includegraphics[width=0.8\textwidth]{/Users/keshavaprasad/Desktop/ThesisWriting/ThesisImages/jet1_pt.png}
	\caption{The transverse momenta of the first matching jet for hadronically decaying \textit{W} boson for SM (in blue), \texttt{M5-NoCut} (in black), \texttt{M5-3TeVCut} (in red) and \texttt{M5-1TeVCut} (in orange) samples, with only the statistical uncertainties being displayed. The last bin contains overflow.}
	\label{Jet1_pt}
\end{figure}

\begin{figure}[!h]%{0.65\textwidth}
	\centering
	\includegraphics[width=0.8\textwidth]{/Users/keshavaprasad/Desktop/ThesisWriting/ThesisImages/jet2_pt.png}
	\caption{The transverse momenta of the second matching jet for hadronically decaying \textit{W} boson for SM (in blue), \texttt{M5-NoCut} (in black), \texttt{M5-3TeVCut} (in red) and \texttt{M5-1TeVCut} (in orange) samples, with only the statistical uncertainties being displayed.The last bin contains overflow. }
	\label{Jet2-pt}
\end{figure}
%\caption{Matching Jets pT}
\begin{figure}[!htb]%{0.65\textwidth}
	\centering
	\includegraphics[width=0.8\textwidth]{/Users/keshavaprasad/Desktop/ThesisWriting/ThesisImages/Jet1_Mass.png}
	\caption{The invariant mass of the first matching jet for hadronically decaying \textit{W} boson for SM (in blue), \texttt{M5-NoCut} (in black), \texttt{M5-3TeVCut} (in red) and \texttt{M5-1TeVCut} (in orange) samples, with only the statistical uncertainties being displayed. The last bin contains overflow.}
	\label{Jet1-Mass}
\end{figure}

\begin{figure}[!h]%{0.65\textwidth}
	\centering
	\includegraphics[width=0.8\textwidth]{/Users/keshavaprasad/Desktop/ThesisWriting/ThesisImages/Jet2_Mass.png}
	\caption{The invariant mass of the second matching jet for hadronically decaying \textit{W} boson for SM (in blue), \texttt{M5-NoCut} (in black), \texttt{M5-3TeVCut} (in red) and \texttt{M5-1TeVCut} (in orange) samples, with only the statistical uncertainties being displayed. The last bin contains overflow.}
	\label{Jet2_Mass}
\end{figure}


\begin{figure}[!htb]%{0.65\textwidth}
	\centering
	\includegraphics[width=0.8\textwidth]{/Users/keshavaprasad/Desktop/ThesisWriting/ThesisImages/Zlep1_pt.png}
	\caption{The transverse momenta of the first matching same flavor, opposite sign lepton for leptonically decaying \textit{Z} boson for SM (in blue), \texttt{M5-NoCut} (in black), \texttt{M5-3TeVCut} (in red) and \texttt{M5-1TeVCut} (in orange) samples, with only the statistical uncertainties being displayed. The last bin contains overflow.}
	\label{Zlep1_pt}
\end{figure}

\begin{figure}[!h]%{0.65\textwidth}
	\centering
	\includegraphics[width=0.8\textwidth]{/Users/keshavaprasad/Desktop/ThesisWriting/ThesisImages/Zlep2_pt.png}
	\caption{The transverse momenta of the second matching same flavor, opposite sign lepton for leptonically decaying \textit{Z} boson for SM (in blue), \texttt{M5-NoCut} (in black), \texttt{M5-3TeVCut} (in red) and \texttt{M5-1TeVCut} (in orange) samples, with only the statistical uncertainties being displayed. The last bin contains overflow.}
	\label{Zlep2-pt}
\end{figure}

\begin{figure}[!h]
	\centering
	\includegraphics[width=0.8\textwidth]{/Users/keshavaprasad/Desktop/ThesisWriting/ThesisImages/Wlep_pt_l_nu.png}
	\caption{The transverse momenta of the other remaining lepton for leptonically decaying \textit{W} boson for SM (in blue), \texttt{M5-NoCut} (in black), \texttt{M5-3TeVCut} (in red) and \texttt{M5-1TeVCut} (in orange) samples, with only the statistical uncertainties being displayed. The last bin contains overflow.}
	\label{Wlep-pT}
\end{figure}

\section{Sensitivity of \textit{WWZ} to Other Operators}\label{AdditionalPlots}

The samples for M2 operator is defined as follows (similar to Table \ref{table4}):

\begin{table}
	\centering
	\begin{tabular}{|c|c|c|c|}
		\hline
		Sample & Kinematic Cut & Color (In Figures) & Definition (in text)\\
		\hline \hline
		SM & No Cut & Blue & \texttt{SM}\\
		\hline
		M2 & No Cut & Black & \texttt{M2-NoCut} \\
		\hline
		M2 & 3 TeV & Red & \texttt{M2-3TeVCut}\\
		\hline
		M2 & 1 TeV & Orange & \texttt{M2-1TeVCut} \\
		\hline
	\end{tabular}
	\caption{The table defining the samples for M2 operator after imposing kinematic cuts.}
	\label{table5}
\end{table}
\begin{figure}[!htb]
	\centering
	\includegraphics[width=0.8\textwidth]{/Users/keshavaprasad/Desktop/Plotting/FinalPlots/M2/HT.png}
	\caption{$H_{T}$ for SM (in blue), \texttt{M2-NoCut} (in black), \texttt{M2-3TeVCut} (in red) and \texttt{M2-1TeVCut} (in orange)  samples, with only the statistical uncertainties being displayed. The last bin contains the overflow. The events are normalized to one.}
	\label{M2-HT}
\end{figure}

\begin{figure}[!htb]
	\centering
	\includegraphics[width=0.8\textwidth]{/Users/keshavaprasad/Desktop/Plotting/FinalPlots/M2/Ntotev.png}
	\caption{The total number of events in the first bin, events passing jet constraints in the second bin and after lepton requirements in the third bin for \texttt{SM} (in blue), \texttt{M2-NoCut} (in black), \texttt{M2-3TeVCut} (in red) and \texttt{M2-1TeVCut} (in orange) samples, with only the statistical uncertainties being displayed.}
	\label{M2Cut flow Histogram-PostCuts}
\end{figure}

\begin{figure}[!htb]%{0.55\textwidth}
	\centering
	\includegraphics[width=0.8\textwidth]{/Users/keshavaprasad/Desktop/Plotting/FinalPlots/M2/Njets.png}
	\caption{The number of jets after requiring at least one jet and $p_{T,jet}>25 \text{ GeV}$  for \texttt{SM} (in blue), \texttt{M2-NoCut} (in black), \texttt{M2-3TeVCut} (in red) and \texttt{M2-1TeVCut} (in orange) samples, with only the statistical uncertainties being displayed. The events are normalized to one.}
	\label{M2NJets-PostCuts}
\end{figure}

\begin{figure}[!htb]
	\centering
	\includegraphics[width=0.8\textwidth]{/Users/keshavaprasad/Desktop/Plotting/FinalPlots/M2/singlejet_mass.png}
	\caption{The invariant mass of a single-jet after requiring $p_{T,jet}>25 \text{ GeV}$ for \texttt{SM} (in blue), \texttt{M2-NoCut} (in black), \texttt{M2-3TeVCut} (in red) and \texttt{M2-1TeVCut} (in orange) samples, with only the statistical uncertainties being displayed. The last bin contains overflow. The events are normalized to one.}
	\label{M2Single Jet Mass -Post Cuts}
\end{figure}

\begin{figure}[!htb]%{0.65\textwidth}
	\centering
	\includegraphics[width=0.8\textwidth]{/Users/keshavaprasad/Desktop/Plotting/FinalPlots/M2/Z_m_l_l.png}
	\caption{The invariant mass of all possible pairs of opposite-sign-same-flavored leptons, for \texttt{SM} (in blue), \texttt{M2-NoCut} (in black), \texttt{M2-3TeVCut} (in red) and \texttt{M2-1TeVCut} (in orange) samples, with only the statistical uncertainties being displayed. The events are normalized to one.}
	\label{M2-Z Candidate}
\end{figure}

\begin{figure}[!htb]
	\centering
	\includegraphics[width=0.8\textwidth]{/Users/keshavaprasad/Desktop/Plotting/FinalPlots/M2/Whad_pt_q_q.png}
	\caption{The transverse momenta of the candidates for hadronically decaying \textit{W} boson for \texttt{SM} (in blue), \texttt{M2-NoCut} (in black), \texttt{M2-3TeVCut} (in red) and \texttt{M2-1TeVCut} (in orange) samples, with only the statistical uncertainties being displayed. The last bin contains overflow. The events are normalized to one.}
	\label{M2-Whad-pT}
\end{figure}

\begin{figure}[!htb]
	\centering
	\includegraphics[width=0.8\textwidth]{/Users/keshavaprasad/Desktop/Plotting/FinalPlots/M2/Wlep_pt_l_nu.png}
	\caption{The transverse momenta of the candidates for leptonically decaying \textit{W} boson for \texttt{SM} (in blue), \texttt{M2-NoCut} (in black), \texttt{M2-3TeVCut} (in red) and \texttt{M2-1TeVCut} (in orange) samples, with only the statistical uncertainties being displayed. The last bin contains overflow. The events are normalized to one.}
	\label{M2-WLep-pT}
\end{figure}

\begin{figure}[!htb]
	\centering
	\includegraphics[width=0.8\textwidth]{/Users/keshavaprasad/Desktop/Plotting/FinalPlots/M2/Z_pt_l_l.png}
	\caption{The transverse momenta of the candidates for leptonically decaying \textit{Z} boson for \texttt{SM} (in blue), \texttt{M2-NoCut} (in black), \texttt{M2-3TeVCut} (in red) and \texttt{M2-1TeVCut} (in orange) samples, with only the statistical uncertainties being displayed. The last bin contains overflow. The events are normalized to one.}
	\label{M2-ZlepleppT}
\end{figure}

\begin{figure}[!h]%{0.65\textwidth}
	\centering
	\includegraphics[width=0.8\textwidth]{/Users/keshavaprasad/Desktop/Plotting/FinalPlots/M2/Z_dPhi_l_l.png}
	\caption{The $\Delta\phi$ between between pair of opposite sign same flavor leptons for \texttt{SM} (in blue), \texttt{M2-NoCut} (in black), \texttt{M2-3TeVCut} (in red) and \texttt{M2-1TeVCut} (in orange) samples, with only the statistical uncertainties being displayed. The events are normalized to one.}
	\label{M2-ZleplepdPhi-PostCuts}
\end{figure}

\begin{figure}[!h]%{0.8\textwidth}
	\centering
	\includegraphics[width=0.8\textwidth]{/Users/keshavaprasad/Desktop/Plotting/FinalPlots/M2/Z_dEta_l_l.png}
	\caption{The $\Delta\eta$ between between pair of opposite sign same flavor leptons for \texttt{SM} (in blue), \texttt{M5-NoCut} (in black), \texttt{M5-3TeVCut} (in red) and \texttt{M5-1TeVCut} (in orange) samples, with only the statistical uncertainties being displayed. The events are normalized to one.}
	\label{M2-ZleplepdEta-PostCuts}
\end{figure}

\begin{figure}[!h]%{0.8\textwidth}
	\centering
	\includegraphics[width=0.8\textwidth]{/Users/keshavaprasad/Desktop/Plotting/FinalPlots/M2/Z_dR_l_l.png}
	\caption{The $\Delta \text{R}$ between between pair of opposite sign same flavor leptons for \texttt{SM} (in blue), \texttt{M5-NoCut} (in black), \texttt{M5-3TeVCut} (in red) and \texttt{M5-1TeVCut} (in orange) samples, with only the statistical uncertainties being displayed. The events are normalized to one.}
	\label{M2-Z-dR-PostCuts}
\end{figure}

\section{Sensitivity of \textit{WWZ} from T5 Operator}
\begin{table}
	\centering
	\begin{tabular}{|c|c|c|c|}
		\hline
		Sample & Kinematic Cut & Color (In Figures) & Definition (in text)\\
		\hline \hline
		SM & No Cut & Blue & \texttt{SM}\\
		\hline
		T5 & No Cut & Black & \texttt{T5-NoCut} \\
		\hline
		T5 & 3 TeV & Red & \texttt{T5-3TeVCut}\\
		\hline
		T5 & 1 TeV & Orange & \texttt{T5-1TeVCut} \\
		\hline
	\end{tabular}
	\caption{The table defining the samples for T5 operator after imposing kinematic cuts.}
	\label{table6}
\end{table}
\begin{figure}[!htb]
	\centering
	\includegraphics[width=0.8\textwidth]{/Users/keshavaprasad/Desktop/Plotting/FinalPlots/T5/HT.png}
	\caption{$H_{T}$ for SM (in blue), \texttt{T5-NoCut} (in black), \texttt{T5-3TeVCut} (in red) and \texttt{T5-1TeVCut} (in orange)  samples, with only the statistical uncertainties being displayed. The last bin contains the overflow. The events are normalized to one.}
	\label{T5-HT}
\end{figure}

\begin{figure}[!htb]
	\centering
	\includegraphics[width=0.8\textwidth]{/Users/keshavaprasad/Desktop/Plotting/FinalPlots/T5/Ntotev.png}
	\caption{The total number of events in the first bin, events passing jet constraints in the second bin and after lepton requirements in the third bin for \texttt{SM} (in blue), \texttt{T5-NoCut} (in black), \texttt{T5-3TeVCut} (in red) and \texttt{T5-1TeVCut} (in orange) samples, with only the statistical uncertainties being displayed.}
	\label{T5Cut flow Histogram-PostCuts}
\end{figure}

\begin{figure}[!htb]%{0.55\textwidth}
	\centering
	\includegraphics[width=0.8\textwidth]{/Users/keshavaprasad/Desktop/Plotting/FinalPlots/T5/Njets.png}
	\caption{The number of jets after requiring at least one jet and $p_{T,jet}>25 \text{ GeV}$  for \texttt{SM} (in blue), \texttt{T5-NoCut} (in black), \texttt{T5-3TeVCut} (in red) and \texttt{T5-1TeVCut} (in orange) samples, with only the statistical uncertainties being displayed. The events are normalized to one.}
	\label{T5NJets-PostCuts}
\end{figure}

\begin{figure}[!htb]
	\centering
	\includegraphics[width=0.8\textwidth]{/Users/keshavaprasad/Desktop/Plotting/FinalPlots/T5/singlejet_mass.png}
	\caption{The invariant mass of a single-jet after requiring $p_{T,jet}>25 \text{ GeV}$ for \texttt{SM} (in blue), \texttt{T5-NoCut} (in black), \texttt{T5-3TeVCut} (in red) and \texttt{T5-1TeVCut} (in orange) samples, with only the statistical uncertainties being displayed. The last bin contains overflow. The events are normalized to one.}
	\label{T5Single Jet Mass -Post Cuts}
\end{figure}

\begin{figure}[!htb]%{0.65\textwidth}
	\centering
	\includegraphics[width=0.8\textwidth]{/Users/keshavaprasad/Desktop/Plotting/FinalPlots/T5/Z_m_l_l.png}
	\caption{The invariant mass of all possible pairs of opposite-sign-same-flavored leptons, for \texttt{SM} (in blue), \texttt{T5-NoCut} (in black), \texttt{T5-3TeVCut} (in red) and \texttt{T5-1TeVCut} (in orange) samples, with only the statistical uncertainties being displayed. The events are normalized to one.}
	\label{T5-Z Candidate}
\end{figure}

\begin{figure}[!htb]
	\centering
	\includegraphics[width=0.8\textwidth]{/Users/keshavaprasad/Desktop/Plotting/FinalPlots/T5/Whad_pt_q_q.png}
	\caption{The transverse momenta of the candidates for hadronically decaying \textit{W} boson for \texttt{SM} (in blue), \texttt{T5-NoCut} (in black), \texttt{T5-3TeVCut} (in red) and \texttt{T5-1TeVCut} (in orange) samples, with only the statistical uncertainties being displayed. The last bin contains overflow. The events are normalized to one.}
	\label{T5-Whad-pT}
\end{figure}

\begin{figure}[!htb]
	\centering
	\includegraphics[width=0.8\textwidth]{/Users/keshavaprasad/Desktop/Plotting/FinalPlots/T5/Wlep_pt_l_nu.png}
	\caption{The transverse momenta of the candidates for leptonically decaying \textit{W} boson for \texttt{SM} (in blue), \texttt{T5-NoCut} (in black), \texttt{T5-3TeVCut} (in red) and \texttt{T5-1TeVCut} (in orange) samples, with only the statistical uncertainties being displayed. The last bin contains overflow. The events are normalized to one.}
	\label{T5-WLep-pT}
\end{figure}

\begin{figure}[!htb]
	\centering
	\includegraphics[width=0.8\textwidth]{/Users/keshavaprasad/Desktop/Plotting/FinalPlots/T5/Z_pt_l_l.png}
	\caption{The transverse momenta of the candidates for leptonically decaying \textit{Z} boson for \texttt{SM} (in blue), \texttt{T5-NoCut} (in black), \texttt{T5-3TeVCut} (in red) and \texttt{T5-1TeVCut} (in orange) samples, with only the statistical uncertainties being displayed. The last bin contains overflow. The events are normalized to one.}
	\label{T5-ZlepleppT}
\end{figure}

\begin{figure}[!h]%{0.65\textwidth}
	\centering
	\includegraphics[width=0.8\textwidth]{/Users/keshavaprasad/Desktop/Plotting/FinalPlots/T5/Z_dPhi_l_l.png}
	\caption{The $\Delta\phi$ between between pair of opposite sign same flavor leptons for \texttt{SM} (in blue), \texttt{T5-NoCut} (in black), \texttt{T5-3TeVCut} (in red) and \texttt{T5-1TeVCut} (in orange) samples, with only the statistical uncertainties being displayed. The events are normalized to one.}
	\label{T5-ZleplepdPhi-PostCuts}
\end{figure}

\begin{figure}[!h]%{0.8\textwidth}
	\centering
	\includegraphics[width=0.8\textwidth]{/Users/keshavaprasad/Desktop/Plotting/FinalPlots/T5/Z_dEta_l_l.png}
	\caption{The $\Delta\eta$ between between pair of opposite sign same flavor leptons for \texttt{SM} (in blue), \texttt{T5-NoCut} (in black), \texttt{T5-3TeVCut} (in red) and \texttt{T5-1TeVCut} (in orange) samples, with only the statistical uncertainties being displayed. The events are normalized to one.}
	\label{T5-ZleplepdEta-PostCuts}
\end{figure}

\begin{figure}[!h]%{0.8\textwidth}
	\centering
	\includegraphics[width=0.8\textwidth]{/Users/keshavaprasad/Desktop/Plotting/FinalPlots/T5/Z_dR_l_l.png}
	\caption{The $\Delta \text{R}$ between between pair of opposite sign same flavor leptons for \texttt{SM} (in blue), \texttt{T5-NoCut} (in black), \texttt{T5-3TeVCut} (in red) and \texttt{T5-1TeVCut} (in orange) samples, with only the statistical uncertainties being displayed. The events are normalized to one.}
	\label{T5-Z-dR-PostCuts}
\end{figure}
